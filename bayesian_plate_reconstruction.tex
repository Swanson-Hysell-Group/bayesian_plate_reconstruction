%% 
%% Copyright 2007, 2008, 2009 Elsevier Ltd
%% 
%% This file is part of the 'Elsarticle Bundle'.
%% ---------------------------------------------
%% 
%% It may be distributed under the conditions of the LaTeX Project Public
%% License, either version 1.2 of this license or (at your option) any
%% later version.  The latest version of this license is in
%%    http://www.latex-project.org/lppl.txt
%% and version 1.2 or later is part of all distributions of LaTeX
%% version 1999/12/01 or later.
%% 
%% The list of all files belonging to the 'Elsarticle Bundle' is
%% given in the file `manifest.txt'.
%% 
%% Template article for Elsevier's document class `elsarticle'
%% with harvard style bibliographic references
%% SP 2008/03/01

\documentclass[preprint,12pt,authoryear]{elsarticle}

%% Use the option review to obtain double line spacing
%% \documentclass[authoryear,preprint,review,12pt]{elsarticle}

%% Use the options 1p,twocolumn; 3p; 3p,twocolumn; 5p; or 5p,twocolumn
%% for a journal layout:
%% \documentclass[final,1p,times,authoryear]{elsarticle}
%% \documentclass[final,1p,times,twocolumn,authoryear]{elsarticle}
%% \documentclass[final,3p,times,authoryear]{elsarticle}
%% \documentclass[final,3p,times,twocolumn,authoryear]{elsarticle}
%% \documentclass[final,5p,times,authoryear]{elsarticle}
%% \documentclass[final,5p,times,twocolumn,authoryear]{elsarticle}

%% For including figures, graphicx.sty has been loaded in
%% elsarticle.cls. If you prefer to use the old commands
%% please give \usepackage{epsfig}

%% The amssymb package provides various useful mathematical symbols
\usepackage{amssymb}
\usepackage{amsmath}
\usepackage{color, soul}
\usepackage{url}
%% The amsthm package provides extended theorem environments
%% \usepackage{amsthm}

%% The lineno packages adds line numbers. Start line numbering with
%% \begin{linenumbers}, end it with \end{linenumbers}. Or switch it on
%% for the whole article with \linenumbers.
\usepackage{lineno}

%% Block of code for fixing corresponding author bug 
%% in elsarticle template... Don't really understand it
\makeatletter
\def\@author#1{\g@addto@macro\elsauthors{\normalsize%
    \def\baselinestretch{1}%
    \upshape\authorsep#1\unskip\textsuperscript{%
      \ifx\@fnmark\@empty\else\unskip\sep\@fnmark\let\sep=,\fi
      \ifx\@corref\@empty\else\unskip\sep\@corref\let\sep=,\fi
      }%
    \def\authorsep{\unskip,\space}%
    \global\let\@fnmark\@empty
    \global\let\@corref\@empty  %% Added
    \global\let\sep\@empty}%
    \@eadauthor={#1}
}
\makeatother
%% End block of weird code


%% Command for bold Greek symbols
\newcommand{\mitbf}[1]{\hbox{\mathversion{bold}$#1$}}

\journal{Earth and Planetary Science Letters}

\begin{document}

\begin{frontmatter}

%% Title, authors and addresses

%% use the tnoteref command within \title for footnotes;
%% use the tnotetext command for theassociated footnote;
%% use the fnref command within \author or \address for footnotes;
%% use the fntext command for theassociated footnote;
%% use the corref command within \author for corresponding author footnotes;
%% use the cortext command for theassociated footnote;
%% use the ead command for the email address,
%% and the form \ead[url] for the home page:
%% \title{Title\tnoteref{label1}}
%% \tnotetext[label1]{}
%% \author{Name\corref{cor1}\fnref{label2}}
%% \ead{email address}
%% \ead[url]{home page}
%% \fntext[label2]{}
%% \cortext[cor1]{}
%% \address{Address\fnref{label3}}
%% \fntext[label3]{}

\title{Bayesian inversion for paleomagnetic reconstruction and plate kinematics}

%% use optional labels to link authors explicitly to addresses:
%% \author[label1,label2]{}
%% \address[label1]{}
%% \address[label2]{}

\author{Ian Rose\corref{cor1}\fnref{ref1}}
\author{Bruce Buffett\fnref{ref1}}
\author{N.L. Swanson-Hysell\fnref{ref1}}

\fntext[ref1]{University of California, Berkeley}
\cortext[cor1]{Corresponding author, \url{ian.rose@berkeley.edu}}

\address{}

\begin{abstract}
%% Text of abstract

\end{abstract}

\begin{keyword}
%% keywords here, in the form: keyword \sep keyword

%% PACS codes here, in the form: \PACS code \sep code

%% MSC codes here, in the form: \MSC code \sep code
%% or \MSC[2008] code \sep code (2000 is the default)

\end{keyword}

\end{frontmatter}

\linenumbers

%% main text
\section{Introduction}
\label{sec:introduction}

\section{Interpretation of APW paths}
\label{sec:pep}

\subsection{Latitudinal drift}
\subsection{Running means and spline fits}
\subsection{Paleomagnetic Euler poles}
\citet{gordon1984paleomagnetic}

\section{Bayesian inversion}
\label{sec:bayesian_inversion}
\subsection{Bayes theorem}
\subsection{Distributions on a sphere}
\subsubsection{Uniform distribution}
\begin{equation}
  \rho(\psi, \phi) = \frac{1}{4 \pi}
\end{equation}
\subsubsection{Fisher distribution}
\begin{equation}
  \begin{aligned}
  \rho(\psi, \phi) 
  &= \frac{1}{C_F} \exp \left( \kappa_F \hat{\mathbf{x}}^T \hat{\mitbf{\mu}} \right) \\
  &= \frac{1}{C_F} \exp \left( \kappa_F \cos \theta \right)
  \end{aligned}
\end{equation}
\begin{equation}
  C_F = \frac{\kappa_F}{4 \pi \sinh{\kappa}}
\end{equation}
\subsubsection{Watson distribution}
\begin{equation}
  \begin{aligned}
  \rho(\psi, \phi) 
  &= \frac{1}{C_W} \exp \left( \kappa_W (\hat{\mathbf{x}}^T \hat{\mitbf{\mu}})^2 \right) \\
  &= \frac{1}{C_W} \exp \left( \kappa_W \cos^2 \theta \right)
  \end{aligned}
\end{equation}
\begin{equation}
  C_W = \left[ {}_1 F_1 \left( \frac{1}{2}, \frac{3}{2}, \kappa_W \right) \right]^{-1}
\end{equation}
\subsection{Markov chain Monte Carlo methods}

\section{Choice of priors}
\label{sec:priors}
\subsection{Euler poles}
\subsection{Ages}

\section{Model selection}
\label{sec:model_selection}

\section{Example inversions}
\label{sec:example_inversion}
\subsection{One stage pole}
\subsection{Two stage poles}
\subsection{Incorporating age uncertainty}

\section{Application to the Keweenawan province}
\label{sec:keweenawan}
\subsection{Geologic context}
\citet{swanson2009no}
\subsection{Inversion for paleomagnetic Euler poles}
\subsection{Plate speeds for Mesoproterozoic Laurentia}

\section{Conclusions}
\label{sec:conclusions}



%% The Appendices part is started with the command \appendix;
%% appendix sections are then done as normal sections
%% \appendix

%% \section{}
%% \label{}

%% If you have bibdatabase file and want bibtex to generate the
%% bibitems, please use
%%
\bibliographystyle{elsarticle-harv} 
\bibliography{bayesian_plate_reconstruction}

%% else use the following coding to input the bibitems directly in the
%% TeX file.

\end{document}

\endinput
%%
%% End of file `elsarticle-template-harv.tex'.
